\documentclass[a4paper]{article}

\usepackage[utf8]{inputenc}
\usepackage[russian]{babel}

\usepackage{mathtext}
\usepackage[T2A]{fontenc}
%\DeclareSymbolFont{T2Aletters}{T2A}{cmr}{m}{it}


\begin{document}

\begin{titlepage}

	\begin{center}

		\large Федеральное государственное автономное образовательное учреждение высшего образования \\
		\large «Санкт-Петербургский политехнический университет Петра Великого» \\
		\large Институт компьютерных наук и технологий \\
		\large Кафедра «Компьютерные интеллектуальные технологии» \\[4cm]

		\huge {\bf Курсовой проект} \\[0.5cm]
		\large {\bf Алгоритмы поиска гамильтонова цикла (сравнительный обзор с примерами программных реализаций)} \\[0.1cm]
		\large по дисциплине «Структуры и алгоритмы компьютерной обработки данных» \\[4cm]

	\end{center}

    \begin{center}
        \begin{minipage}[t]{4cm}
            \begin{flushleft}
                Выполнил студент гр. 23506/1
            \end{flushleft}
        \end{minipage}
        \hfill
        \begin{minipage}[t]{4cm}
            \begin{flushright}
            О.Д. Романов
            \end{flushright}
        \end{minipage} \\[0.5cm]

        \begin{minipage}[t]{4cm}
            \begin{flushleft}
                Руководитель доцент, к.ф.-м.н.
            \end{flushleft}
            \flushleft
        \end{minipage}
        \hfill
        \begin{minipage}[t]{4cm}
            \begin{flushright}
                В.Г. Пак
            \end{flushright}
        \end{minipage}
    \end{center}

    \begin{flushright}
        23 марта 2017
    \end{flushright}

	
	\vfill

	\begin{center}
	    \large Санкт-Петербург\\
	    \large \the\year
	\end{center}

\thispagestyle{empty}
\end{titlepage}

\tableofcontents
\newpage

\section{Постановка задачи}

\subsection{Определения}
Гамильтоновой цепью графа называется его простая цепь, которая проходит через каждую вершину графа точно один раз.
Цикл графа, проходящий через каждую его вершину, называется гамильтоновым циклом.
Граф называется гамильтоновым, если он обладает гамильтоновым циклом.

Указанные названия цепей и циклов связаны с именем Уильяма Гамильтона (Hamilton W.), который в 1859 году предложил следующую игру-головоломку:
требуется, переходя по очереди от одной вершины додекаэдра к другой вершине по его ребру, обойти все 20 вершин по одному разу и вернуться в начальную вершину.

\subsection{Другие задачи о поиске гамильтонова цикла}
Придумано еще много других развлекательных и полезных задач, связанных с поиском гамильтоновых циклов.

\subsubsection{Задача про банкет}
Компанию из нескольких человек требуется рассадить за круглым столом таким образом, чтобы по обе стороны от каждого сидели его знакомые.
Очевидно, для решения этой задачи нужно найти гамильтонов цикл в графе знакомств компании.

\subsubsection{Задача о шахматном коне}
Можно ли, начиная с произвольного поля шахматной доски, обойти конем последовательно все 64 поля по одному разу и вернуться в исходное поле?


\section{Практическая значимость задачи}
Задача имеет воплощение в реальной жизни. Допустим, курьеру нужно доставить несколько заказов по точкам.
Нудно составить оптимальный маршрут для курьера.
Задача коммивояжера.


\end{document}